\documentclass[12 pt, twoside] {article}
\usepackage[margin=1in]{geometry}
\usepackage[utf8]{inputenc}
\usepackage{listings}
\usepackage{color}
\usepackage{textcomp}
\usepackage{setspace}
\usepackage{verbatim}
\usepackage{graphicx}
\usepackage{footnote}
\usepackage{enumitem}
\usepackage{fancyhdr}
\usepackage[parfill]{parskip}

\makesavenoteenv{tabular}
\makesavenoteenv{table}

\setlist[itemize]{noitemsep, topsep=0pt}

\newcommand\la{\textlangle}
\newcommand\ra{\textrangle}

\setlength{\parindent}{0pt}

\definecolor{codegreen}{rgb}{0,0.6,0}
\definecolor{codegray}{rgb}{0.5,0.5,0.5}
\definecolor{codeblue}{rgb}{0,0,0.6}
\definecolor{backcolor}{rgb}{0.95,0.95,0.95}

\lstdefinestyle{mystyle}{
	backgroundcolor = \color{backcolor},
	commentstyle = \color{codeblue},
	keywordstyle = \color{codegreen},
	numberstyle = \color{codegray},
	stringstyle = \color{magenta},
	basicstyle = \footnotesize\ttfamily,
	breakatwhitespace = false,
	breaklines = true,
	captionpos = b,
	keepspaces = true,
	numbers = left,
	numbersep = 5pt,
	showspaces = false,
	showstringspaces = false,
	showtabs = false,
	tabsize = 4
}

\lstset{style = mystyle}

\pagestyle{fancy}
\fancyhf{}
\rhead{Yicheng Wang}
\lhead{Graph Theory 2}

\begin{document}
{\catcode`?=\active
\def?!#1!{\footnote{#1}}
\section*{MST \& Shortest Path}

Now that we all know what a graph is, its basic properties, and how to traverse
them. Now let's talk about some more specific features of graphs.

\subsection*{Minimum Spanning Tree (MST)}

A tree is a perennial plant with an elongated stem, or trunk, supporting... Oh
wait, wrong wikipedia page.

A tree in CS is a \textbf{connected}, \textbf{acyclic} graph. A spanning tree in
a graph is sub-graph that is a tree that connects all vertices within a given graph.
A \textbf{Minimum Spanning Tree (MST)} is a spanning tree with the minimum total
edge weight.

In English, a MST is a subset of the edges that connects the graph with the
minimum total weight. There are various real life applications of minimum
spanning tree, such as choosing how to lay optic cable to connect a bunch of
data centers or roads connecting cities. But in contest, a lot of times you will
just be asked to ``find the MST."

There are two algorithms used, both are pretty intuitive, and have the same
runtime but make sure you understand the essence because sometimes you'll have
to construct the MST without constructing the graph.

\subsubsection*{Prim's Algorithm}

Prim's algorithm starts with an arbitrary vertex, $v_0$, and divide up the graph
into two sets of vertices: $V$ and $U$. Initially $V = \{v_0\}$ and $U$ is
everything else in the graph. While $U$ is non-empty / $V \neq G$, we pick the
edge $(v, u)$ with $v \in V$ and $u \in U$ with the minimum edge weight. Add the
edge to the MST, and move $u$ form $U$ to $V$. In psudocode, it is:
\begin{verbatim}
Prim(Graph G):
    boolean visited[V] initialized to false
    visited[v_0] = true                 // for arbitrary v_0
    MST = {}

    while not all visited:
        e = (v, u) with minimum edge weight
            where visited[v] = true and visited[u] = false
        MST.add(e)
        visited[u] = true

    return MST
\end{verbatim}

The runtime of this largely depends on finding the edge with minimum weight
between the visited and unvisited set. If we conduct a linear search, the
runtime is $O(V^2)$ whereas if we use a priority queue it is $O(E \log(V))$.

\subsubsection*{Kruskal's Algorithm}

Kruskal's algorithm is the opposite of Prim's. Instead of building a tree
spawning from one vertex. It builds a forest that eventually becomes one single
spanning tree. This algorithm is greedy in nature as well. We began by picking
the shortest edge, adding that to the MST, then we combine the two trees at the
endpoints of the edge. In psudocode it is:
\begin{verbatim}
Kruskal(Graph G):
    MST = {}
    set trees[V] initialized to {v} for each v in V

    for e = (u, v) in G.E ordered by weight increasingly:
        if trees[u] =/= trees[v]:
            MST.add(e)
            trees[u] = trees[v] = UNION(trees[u], trees[v])

    return MST
\end{verbatim}

Note that the runtime of this is exactly like the runtime of Prim $O(E \log(V))$
because we literally go through each edge and the unions of sets can be done in
$O(\log(N))$ time where $N$ is the size of the larger set.

\subsection*{Shortest Path -- Dijkstra}

As I have said last time, there is a reason why DFS and BFS should be renamed to
depth first transversal and breath first transversal because there are better
searching algorithms. And today we're going to learn about the better algorithm
-- Dijkstra's. The advantage of this over the other two is that this algorithm
is targeted in finding the shortest path between two points in a graph, and have
significantly better run time. Dijkstra's algorithm is a modified version of
what is known as \textbf{best first search}.

We assign a distance to each node, 0 at the initial node and infinity at every
other node. Then at each node, we calculate the tentative distance to each of
its neighbors by adding the weight of the edge to the distance at the current
node. If the tentative distance is smaller than the distance currently
associated with the neighbor, we will update the distance to the smaller value.
We will then mark the current node as visited and move to the neighbor with the
lowest associated distance.

This algorithm is a bit more complex than the others when written in psudocode
but here is what it should look like:
\begin{verbatim}
Dijkstra(Graph G, Vertex src, Vertex dest):
    boolean visited[V] initialized to false
    int dist[V] initialized to infinity
    priorityQueue frontier sorted by dist, with min on top

    dist[src] = 0
    frontiner.push(src)

    while (!frontier.empty()):
        v = frontier.pop()
        
        if visited[v]:
            continue; // already been here, skip!

        if v == dest:
            return dist[dest]

        for (v, u) in G.E with weight w:
            if dist[u] > dist[v] + w:
                dist[u] = dist[v] + w
                frontier.push(u)

        visited[v] = true

    return dist[dest]
\end{verbatim}

The algorithm terminates when the node we are looking for has been visited, or
when the entire graph has been traversed. The algorithm has the worst-case
runtime of $O(E + V \log(V))$ if we use a priority queue to store distance.

\subsection*{Floyd-Warshall}

Dijkstra's algorithm is one that calculates the shortest distance from one point
to another (and potentially one point to every other point in the graph).
However, sometimes we need to find all pairwise distances between all pairs of
nodes. This can be done with repeated Dijkstra's with time complexity of $O(EV +
V^2 \log(V))$. However, this is doing quite a bit of overcounting as we are
traversing the graph more times than necessary. In truth we only need to
traverse all the edges one per vertex. This is known as the Floyd-Warshall
algorithm. It repeated gets estimates of all pairwise distances and stop at the
optimal one.

In psudocode it is:
\begin{verbatim}
let dist be a 2D array of distances between vertices initialized to infinity
for v from 1...V:
    dist[v][v] = 0
for (u, v) in E with weight w:
    dist[u][v] = w

for k from 1...V:
    for j from 1...V:
        for i from 1...V:
            if dist[i][j] > dist[i][k] + dist[k][j]
                dist[i][j] = dist[i][k] + dist[k][j]
\end{verbatim}

Basically the idea behind this is that if we can go from point i to j faster
through k, we'll do it. Note that this algorithm does not give the exact path, 
a simple modification will return the path. [every time we do the substitution
w/ k, we know it is a part of the path]

\end{document}
