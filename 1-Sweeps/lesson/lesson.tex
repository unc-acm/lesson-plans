\documentclass[12 pt, twoside] {article}
\usepackage[margin=1in]{geometry}
\usepackage[utf8]{inputenc}
\usepackage{listings}
\usepackage{color}
\usepackage{textcomp}
\usepackage{setspace}
\usepackage{verbatim}
\usepackage{graphicx}
\usepackage{footnote}
\usepackage{enumitem}
\usepackage{fancyhdr}
\usepackage{tikz}
\usepackage[parfill]{parskip}

\makesavenoteenv{tabular}
\makesavenoteenv{table}

\setlist[itemize]{noitemsep, topsep=0pt}

\newcommand\la{\textlangle}
\newcommand\ra{\textrangle}

\setlength{\parindent}{0pt}

\definecolor{codegreen}{rgb}{0,0.6,0}
\definecolor{codegray}{rgb}{0.5,0.5,0.5}
\definecolor{codeblue}{rgb}{0,0,0.6}
\definecolor{backcolor}{rgb}{0.95,0.95,0.95}

\lstdefinestyle{mystyle}{
	backgroundcolor = \color{backcolor},
	commentstyle = \color{codeblue},
	keywordstyle = \color{codegreen},
	numberstyle = \color{codegray},
	stringstyle = \color{magenta},
	basicstyle = \footnotesize\ttfamily,
	breakatwhitespace = false,
	breaklines = true,
	captionpos = b,
	keepspaces = true,
	numbers = left,
	numbersep = 5pt,
	showspaces = false,
	showstringspaces = false,
	showtabs = false,
	tabsize = 4
}

\newenvironment{blockquote}{%
  \par%
  \medskip
  \leftskip=4em\rightskip=2em%
  \noindent\ignorespaces}{%
  \par\medskip}

\lstset{style = mystyle}

\pagestyle{fancy}
\fancyhf{}
\rhead{Yicheng Wang}
\lhead{Sweeping Algorithms}

\begin{document}
{\catcode`?=\active
\def?!#1!{\footnote{#1}}
\section*{Sweeping Algorithms}

Sweeping algorithms are a common technique used in computational geometry that
relies on the fact that certain physical properties can be ordered to
drastically save time.

\subsection*{Line Sweep}

In this technique, we are usually given a list of segments and asked to prove
something about them. Let us look at an example problem of finding the location
of the maximum overlap between a bunch of line segments:
\begin{blockquote}
    Given a list of segments all on the same line in the form of $l_i, r_i$
    where $l_i$ represents the left endpoint of the $i^{\text{th}}$ segment and
    $r_i$ represents its right endpoint, find the maximum number of lines
    overlapped with each other.
\end{blockquote}

Sweeping algorithms relies on 2 things: a sweep line and a bunch of events. To
solve this problem we imagine a vertical sweep line moving from left to right.
When it encounters an event (either enters a segment on a $l_i$ or leaves a
segment on a $r_i$), we will update the counter of how many lines the vertical
sweep line is crossing at any given point (i.e. how many lines overlap at a
given point), and in that process we can find the maximum number of lines that
overlap with each other.

\begin{center}
\begin{tikzpicture}
    \draw[thick] (-1,0) -- (1,0);
    \draw[thick] (2,0) -- (7,0);
    \draw[thick] (1,-1) -- (5,-1);
    \draw[thick] (6, -1) -- (8, -1);
    \draw[thick] (4,1) -- (6.5,1);

    \draw[red, thick] (3, 2) node[anchor=south] {Current Overlap = 2} -- (3, -2) node[anchor=north] {Sweep Line};
    \draw[red, thick, ->] (3.2,1.5) -- (4,1.5);
\end{tikzpicture}
\end{center}

On an implementation note, the way one would actually go about doing this would
be to turn the line segments into a list of \textbf{events}, which can either
have a role of activate or deactivate (corresponds to $l$ or $r$ index
respectively). We can then sort this list by the coordinates and go trough the
sorted list, which is equivalent to moving the sweep line from left to right.
Every time we encounter an activate event we increment our counter by one, and
every time we encounter a deactivate event we decrement our counter by one. This
way by the end of the list we will know the maximum overlaps. This procedure
takes $O(n \log n + n)$ time, first to sort the list then to access it, so the
overall time is $O(n \log n)$.

A more complex example involves given a group of lines and finding all lines
that intersect. Formally, given a list of $(x_1, y_1, x_2, y_2)$ coordinates
marking the endpoints of several segments, and assuming that no three intersect
at the same point, determine if any of the two lines intersect. For this problem we
will once again use a sweep line form left to right and use left and right
endpoints of segments as events. However, what we do at each event is different.
We keep a list of active segments, which are segments that the sweep line
intersects for any given $x$ value, whenever the sweep line encounters an event,
we first update the set of active segments and sort them by their y-values. In
order for two segments to intersect. The order of their $x$ values must be the
opposite of the order of their $y$ values, and for two to intersect, at some
event they must be adjacent to each other on the ordering by $y$ coordinates.
Therefore our algorithm only needs to check for intersection of the segments
right above and below the segment of the event. This reduces the amount of
checking needed to $O(n \log n)$ instead of $O(n^2)$.

\subsection*{Radial Sweep}

Radial sweeps use a similar concept except instead of having a vertical sweep
line move from left to right we have an rotating sweep line to sweep either
clockwise or counterclockwise. Again the line is imaginary, what we actually do
is have a list of events then sort them based on the angle. This technique can
be used to find the maximum number of collinear points in a set of points. We
can do this based off of the fact that if three points are collinear then from
one of the points, the angle to the other two would be the same. Therefore, to
find the maximum number of collinear points, we go to each point, compute its
angle to every other point, then sort that list (effectively performing a radial
sweep), then we can go through that list and count how many elements are
collinear (we can do this in $n$ time). Therefore we can do this in $O(n^2 \log
n)$ time, which is pretty good.

\end{document}
