\documentclass[12 pt, twoside] {article}
\usepackage[margin=1in]{geometry}
\usepackage[utf8]{inputenc}
\usepackage{listings}
\usepackage{color}
\usepackage{textcomp}
\usepackage{setspace}
\usepackage{verbatim}
\usepackage{graphicx}
\usepackage{footnote}
\usepackage{enumitem}
\usepackage{fancyhdr}
\usepackage{amsmath}
\usepackage[parfill]{parskip}

\makesavenoteenv{tabular}
\makesavenoteenv{table}

\setlist[itemize]{noitemsep, topsep=0pt}

\newcommand\la{\textlangle}
\newcommand\ra{\textrangle}

\setlength{\parindent}{0pt}

\definecolor{codegreen}{rgb}{0,0.6,0}
\definecolor{codegray}{rgb}{0.5,0.5,0.5}
\definecolor{codeblue}{rgb}{0,0,0.6}
\definecolor{backcolor}{rgb}{0.95,0.95,0.95}

\lstdefinestyle{mystyle}{
	backgroundcolor = \color{backcolor},
	commentstyle = \color{codeblue},
	keywordstyle = \color{codegreen},
	numberstyle = \color{codegray},
	stringstyle = \color{magenta},
	basicstyle = \footnotesize\ttfamily,
	breakatwhitespace = false,
	breaklines = true,
	captionpos = b,
	keepspaces = true,
	numbers = left,
	numbersep = 5pt,
	showspaces = false,
	showstringspaces = false,
	showtabs = false,
	tabsize = 4
}

\lstset{style = mystyle}

\pagestyle{fancy}
\fancyhf{}
\rhead{Yicheng Wang}
\lhead{Strongly Connected Components}

\begin{document}
{\catcode`?=\active
\def?!#1!{\footnote{#1}}
\section*{Strongly Connected Components}

\textbf{Def.} We say that vertices $u$ and $v$ are in the same strongly
connected component of a graph if there exists a path from $u$ to $v$ and from
$v$ to $u$.

Notice that in an undirected graph, connectedness implies strongly
connectedness, as every edge is bidirectional. Therefore we are going to only
consider directed graphs.

Finding the strongly connected components of a graph allows us to partition the
graph into strongly connected subgraphs within which one can get from each
vertex to every other vertex.

\subsection*{Kosaraju's algorithm}

Kosaraju's algorithm is what we can use to find the strongly connected
components of the graph in $O(E + V)$ time. The algorithm relies on the fact
that the strongly connected components of the graph are the same as the strongly
connected components of the transpose graph (same graph with the direction of
all the edges reversed).

Kosaraju's algorithm is essentially two DFS, one on the graph and another on the
reverse graph. The first DFS on the original graph is a reverse topological
sort(ish), we sort the list of vertices in such a way that $u$ comes before $v$
if there exists and edge from $v$ to $u$. I say this is approximately a
topological sort because technically one cannot topologically sort a graph if it
contains a cycle, but in our case, if it contains a cycle, we ignore it, and
keep going. Note that this will sometimes create forests in connected graphs as
a result, but that's fine.

Now we have this list, note that the first element of the list is a node in the
original graph that is either in a cycle or has no incoming edges. Now we dfs
from this node on the \textbf{transpose} graph, and every element it can get to
is guaranteed to be in the same strongly connected component as it, this is
because the SCC of the transpose graph is the same as that of the original
graph. After we dfs from that point, we move onto the next un-assigned element
in the sorted list and repeat the process, thereby generating a group of
strongly connected subgraphs.


\subsection*{Boolean Satisfiability Problems}

There is a special type of problems called boolean satisfiability problems that
can be solved with Kosaraju's algorithm. The problem is relatively simple, given
a bunch of boolean variables $x_1, x_2, \dots, x_n$ and a boolean expression
expressed as the conjunction of several pairwise disjunctions using these
variables. Our task is to determine if there exists a solution.

To solve this problem we introduce something called the \textbf{implication
graph}. The implication graph is a graph where there exists an edge between two
nodes if there exists a logical implication that if one node is true so is the
next. We construct the implication graph as follows: if the entire boolean
expression is true, then each of the pairwise disjunctions are true (because
they are all connected via conjunctions). Therefore suppose we have $(x_i \lor
x_j)$ as a part of the conjunction, then we can construct two edges: $\lnot x_i
\to x_j$ and $\lnot x_j \to x_i$.

Once we construct the implication graph, we know that the equation is
\textbf{NOT} satisfiable if $x_i$ and $\lnot x_i$ are within the strongly
connected component. As that would mean $x_i$ and $\lnot x_i$ must both be true.

Otherwise we can pick an arbitrary starting point, assign it \texttt{true}, its
opposite \texttt{false} and floodfill those truth values to its strongly
connected component.

\end{document}
